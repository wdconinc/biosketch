% NSF GPG 16-1 January 25, 2016 
% Sect II.C.2.f
%
% Biographical Sketch(es)
%
% (i)	Senior Personnel
%
% A biographical sketch (limited to two pages) is required for each individual identified as senior personnel. (See GPG Exhibit II-7 for the definitions of Senior Personnel.) Proposers may elect to use third-party solutions, such as NIH’s SciENcv to develop and maintain their biographical sketch. However, proposers are advised that they are still responsible for ensuring that biographical sketches created using third-party solutions are compliant with NSF proposal preparation requirements.
%
% Do not submit any personal information in the biographical sketch. This includes items such as: home address; home telephone, fax, or cell phone numbers; home e-mail address; drivers’ license numbers; marital status; personal hobbies; and the like. Such personal information is not appropriate for the biographical sketch and is not relevant to the merits of the proposal. NSF is not responsible or in any way liable for the release of such material. (See also GPG Chapter III.H).
%
% (a) Professional Preparation
%
% A list of the individual’s undergraduate and graduate education and postdoctoral training (including location) as indicated below:
%
% Undergraduate Institution(s)  Location  Major  Degree & Year
%
% Graduate Institution(s)  Location  Major  Degree & Year
%
% Postdoctoral Institution(s)  Location  Area  Inclusive Dates (years)
%
% (b) Appointments
%
% A list, in reverse chronological order, of all the individual's academic/professional appointments beginning with the current appointment.
%
% (c) Products
%
% A list of:
% (i) up to five products most closely related to the proposed project; and
% (ii) up to five other significant products, whether or not related to the proposed project.
%
% Acceptable products must be citable and accessible including but not limited to publications, data sets, software, patents, and copyrights. Unacceptable products are unpublished documents not yet submitted for publication, invited lectures, and additional lists of products. Only the list of ten will be used in the review of the proposal.
%
% Each product must include full citation information including (where applicable and practicable) names of all authors, date of publication or release, title, title of enclosing work such as journal or book, volume, issue, pages, website and URL or other Persistent Identifier.
%
% If only publications are included, the heading "Publications" may be used for this section of the Biographical Sketch.
%
% (d) Synergistic Activities
% A list of up to five examples that demonstrate the broader impact of the individual’s professional and scholarly activities that focuses on the integration and transfer of knowledge as well as its creation. Examples could include, among others: innovations in teaching and training (e.g., development of curricular materials and pedagogical methods); contributions to the science of learning; development and/or refinement of research tools; computation methodologies, and algorithms for problem-solving; development of databases to support research and education; broadening the participation of groups underrepresented in STEM; and service to the scientific and engineering community outside of the individual’s immediate organization.

\section*{Biographical Sketch: Wouter Deconinck}

\begin{tabular}{@{}ll}
Address & College of William \& Mary, Department of Physics \\
        & P.O. Box 8795, Williamsburg, VA 23187 \\
Phone   & (757) 221-3539 \\
Email   & wdeconinck@wm.edu \\
\end{tabular}

%---------------------------------------------------------------------------

\subsubsection*{Professional preparation}

\begin{tabular}{lll} % allow indentation of columns
\multicolumn{3}{@{}l}{\bf Undergraduate institution:} \\
University of Gent (Belgium) & Engineering Physics & M.S., July 2003 \\
\multicolumn{3}{@{}l}{\bf Graduate institution:} \\
University of Michigan & Physics & Ph.D., April 2008 \\
\multicolumn{3}{@{}l}{\bf Postdoctoral institutions:} \\
Deutsche Electronen-Synchrotron (Germany) & Hadronic Physics & January--May 2008 \\
Massachusetts Institute of Technology & Electroweak Precision Tests & June 2008--June 2010
\end{tabular}

%---------------------------------------------------------------------------

\subsubsection*{Appointments}

\textbf{Assistant professor,} July 2010--present \\
Department of Physics, College of William \& Mary, Williamsburg, VA \\
%
\textbf{Postdoctoral associate,} June 2008--June 2010 \\
Laboratory for Nuclear Science, Massachusetts Institute of Technology, Cambridge, MA \\
%
\textbf{Research fellow} (competitive selection), January--May 2008 \\
Deutsches Elektronen-Synchrotron (DESY), Germany \\
%
\textbf{Research assistant,} July 2003--December 2007 \\
Department of Physics, University of Michigan, Ann Arbor, MI

%---------------------------------------------------------------------------

\subsubsection*{Products closely related to the proposed project\footnotemark{}}

\footnotetext{Due to the size of nuclear and high-energy physics collaborations it is not possible to list all collaborators and affiliations within the constraints of this document. The full author lists are provided in the references.}

\begin{itemize}
 \setlength{\itemsep}{0pt}
 \setlength{\parskip}{0pt}
 \setlength{\parsep}{0pt}
 % InspireHEP, CV format (LaTeX)

 \item%{Magee:2016xqx}
  {\bf ``A novel comparison of M{\o}ller and Compton electron-beam polarimeters''},
  J.~A.~Magee {\it et al.},
  arXiv:1610.06083 [physics.ins-det],
  %JLAB-PHY-16-2333
  %(Oct 19, 2016)
  %\href{http://inspirehep.net/record/1492765}{HEP entry}
  \cite{Magee:2016xqx}.

 \item%{Androic:2013rhu}
  {\bf ``First Determination of the Weak Charge of the Proton''},
  D.~Androic {\it et al.} [Qweak Collaboration],
  arXiv:1307.5275 [nucl-ex],
  %DOI:10.1103/PhysRevLett.111.141803,
  Phys.\ Rev.\ Lett.\  {\bf 111}, no. 14, 141803 (2013),
  %JLAB-PHY-13-1756
  %(Jul 19, 2013)
  %\href{http://inspirehep.net/record/1243627}{HEP entry}
  %76 citations counted in INSPIRE as of 12 Nov 2016
  \cite{Androic:2013rhu}.
  
 \item%{Allison:2014tpu}
  {\bf ``The Q$_{weak}$ experimental apparatus''},
  T.~Allison {\it et al.} [Qweak Collaboration],
  arXiv:1409.7100 [physics.ins-det],
  %DOI:10.1016/j.nima.2015.01.023,
  Nucl.\ Instrum.\ Meth.\ A {\bf 781}, 105 (2015),
  %JLAB-PHY-14-1959
  %(Sep 24, 2014)
  %\href{http://inspirehep.net/record/1318968}{HEP entry}
  %11 citations counted in INSPIRE as of 12 Nov 2016
  %\cite{Airapetian:2014gfp}
  \cite{Allison:2014tpu}.

 \item%{Wang:2014bba}
  {\bf ``Measurement of parity violation in electron–quark scattering''},
  D.~Wang {\it et al.} [PVDIS Collaboration],
  %DOI:10.1038/nature12964,
  Nature {\bf 506}, no. 7486, 67 (2014),
  %(2014)
  %\href{http://inspirehep.net/record/1280371}{HEP entry}
  %26 citations counted in INSPIRE as of 12 Nov 2016
  \cite{Wang:2014bba}.

 \item%{Abrahamyan:2012gp}
  {\bf ``Measurement of the Neutron Radius of 208Pb Through Parity-Violation in Electron Scattering''},
  S.~Abrahamyan {\it et al.},
  arXiv:1201.2568 [nucl-ex],
  %DOI:10.1103/PhysRevLett.108.112502,
  Phys.\ Rev.\ Lett.\  {\bf 108}, 112502 (2012),
  %JLAB-PHY-12-1480
  %(Jan 2012)
  %\href{http://inspirehep.net/record/1084413}{HEP entry}
  %194 citations counted in INSPIRE as of 12 Nov 2016
  \cite{Abrahamyan:2012gp}.

\end{itemize}

%---------------------------------------------------------------------------

\subsubsection*{Other significant products}
\begin{itemize}
 \setlength{\itemsep}{0pt}
 \setlength{\parskip}{0pt}
 \setlength{\parsep}{0pt}
 
 \item%{Abeyratne:2012ah}
  {\bf ``Science Requirements and Conceptual Design for a Polarized Medium Energy Electron-Ion Collider at Jefferson Lab''},
  S.~Abeyratne {\it et al.},
  arXiv:1209.0757 [physics.acc-ph],
  %JLAB-ACC-12-1619
  %(Sep 2012)
  %\href{http://inspirehep.net/record/1184353}{HEP entry}
  %57 citations counted in INSPIRE as of 12 Nov 2016
  \cite{Abeyratne:2012ah}.

 \item%{Qian:2011py}
  {\bf ``Single Spin Asymmetries in Charged Pion Production from Semi-Inclusive Deep Inelastic Scattering on a Transversely Polarized $^3$He Target''},
  X.~Qian {\it et al.} [Jefferson Lab Hall A Collaboration],
  arXiv:1106.0363 [nucl-ex],
  %DOI:10.1103/PhysRevLett.107.072003
  Phys.\ Rev.\ Lett.\  {\bf 107}, 072003 (2011),
  %JLAB-PHY-11-1332
  %(Jun 2011)
  %\href{http://inspirehep.net/record/902486}{HEP entry}
  %144 citations counted in INSPIRE as of 12 Nov 2016
  \cite{PhysRevLett.107.072003}.

 \item%{Ahmed:2011vp}
  {\bf ``New Precision Limit on the Strange Vector Form Factors of the Proton''},
  Z.~Ahmed {\it et al.} [HAPPEX Collaboration];
  arXiv:1107.0913 [nucl-ex];
  %DOI:10.1103/PhysRevLett.108.102001
  Phys.\ Rev.\ Lett.\  {\bf 108}, 102001 (2012);
  %JLAB-PHY-11-1479
  %(Jul 2011)
  %\href{http://inspirehep.net/record/916996}{HEP entry}
  %69 citations counted in INSPIRE as of 12 Nov 2016
  \cite{PhysRevLett.108.102001}.

 \item%{Huang:2011bc}
  {\bf ``Beam-Target Double Spin Asymmetry A\_LT in Charged Pion Production from Deep Inelastic Scattering on a Transversely Polarized He-3 Target at 1.4<Q\textasciicircum{}2<2.7 GeV\textasciicircum{}2''},
  J.~Huang {\it et al.} [Jefferson Lab Hall A Collaboration];
  arXiv:1108.0489 [nucl-ex];
  %DOI:10.1103/PhysRevLett.108.052001
  Phys.\ Rev.\ Lett.\  {\bf 108}, 052001 (2012);
  %JLAB-PHY-11-1359
  %(Aug 2011)
  %\href{http://inspirehep.net/record/921786}{HEP entry}
  %46 citations counted in INSPIRE as of 12 Nov 2016
  \cite{PhysRevLett.108.052001}.

 \item%{Narayan:2015aua}
  {\bf ``Precision Electron-Beam Polarimetry at 1 GeV Using Diamond Microstrip Detectors''},
  A.~Narayan {\it et al.};
  arXiv:1509.06642 [nucl-ex];
  %DOI:10.1103/PhysRevX.6.011013
  Phys.\ Rev.\ X {\bf 6}, no. 1, 011013 (2016);
  %JLAB-PHY-15-2108
  %(Sep 22, 2015)
  %\href{http://inspirehep.net/record/1394432}{HEP entry}
  %3 citations counted in INSPIRE as of 12 Nov 2016
  \cite{Narayan:2015aua}.

\end{itemize}

%---------------------------------------------------------------------------

\subsubsection*{Synergistic activities}
\begin{itemize}
 \setlength{\itemsep}{0pt}
 \setlength{\parskip}{0pt}
 \setlength{\parsep}{0pt}

 \item {\bf W\&M Physics REU Coordinator:} The PI has been the coordinator of the long-running physics REU program at W\&M from 2011 through 2015. Combined with his position on the graduate admissions committee, he has many contacts and collaborators at local colleges (including Norfolk State University, Virginia Union University, Hampton University, Christopher Newport University, and Old Dominion University).

 \item {\bf Co-PI of the Jefferson Lab Promising Young Physicist program} which supports the career development of Jefferson Lab-affiliated postdocs by providing them with opportunities for colloquium presentations and organizing mock job interviews. Deconinck was PI of the program from 2011 through 2015, before transferring the leadership to a former participant.

 \item {\bf Sex and gender diversity in physics:} The PI is a member of LGBT+Physicists, a national group aiming to improve gender and sexual diversity in physics.  He was co-organizer of sessions at the 2012 and 2013 APS March meetings about the climate for sexual minorities. He was co-auther of an influential report on the status of LGBT people in physics\,\cite{APS:LGBT2016}.
 
 \item {\bf Gender \& Race in the Physical Sciences:} The PI has developed and taught (in Fall 2015 and Spring 2016, both times as an overload) a novel course on the impact of gender and race on who participates in the scientific enterprise in the United States. The discussion-based course has been popular primarily with white women and students of color in the sciences. The PI will continue to offer this course regularly in the future. All course materials are available as a open educational resource under a Creative Commons license\,\cite{CC-BY-SA}.
 
 \item {\bf Makerspaces and Physics Innovation and Entrepreneurship (PIE):} Research indicates that retention of underrepresented physics students is improved by providing examples of actual career outcomes. Through the founding of the Small Hall Makerspace (since 2013) and the development of a new curricular track in ``Engineering Physics and Applied Design,'' the PI is active in enabling students to build skills that will be valuable outside academia.

\end{itemize}
