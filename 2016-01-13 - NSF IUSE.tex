\documentclass[11pt,letterpaper]{article}
\usepackage[utf8]{inputenc}

%%%%%%%%%%%%%%%%%%%%%%%%%%%%%%%%%%%%%%%%%%%%%%%%%%%%%%%%%%%%%%%%%%%%%%%%%
\pagestyle{plain}                                                      %%
%%%%%%%%%% EXACT 1in MARGINS %%%%%%%                                   %%
\setlength{\textwidth}{6.5in}     %%                                   %%
\setlength{\oddsidemargin}{0in}   %% (It is recommended that you       %%
\setlength{\evensidemargin}{0in}  %%  not change these parameters,     %%
\setlength{\textheight}{8.5in}    %%  at the risk of having your       %%
\setlength{\topmargin}{0in}       %%  proposal dismissed on the basis  %%
\setlength{\headheight}{0in}      %%  of incorrect formatting!!!)      %%
\setlength{\headsep}{0in}         %%                                   %%
\setlength{\footskip}{.5in}       %%                                   %%
%%%%%%%%%%%%%%%%%%%%%%%%%%%%%%%%%%%%                                   %%
\newcommand{\required}[1]{\section*{\hfil #1\hfil}}                    %%
\renewcommand{\refname}{\hfil References Cited\hfil}                   %%
%%%%%%%%%%%%%%%%%%%%%%%%%%%%%%%%%%%%%%%%%%%%%%%%%%%%%%%%%%%%%%%%%%%%%%%%%

% Bibliography
%%% BibLaTeX %%%
\usepackage[backend=bibtex,natbib,maxcitenames=1,style=numeric,doi=false,isbn=false,url=false]{biblatex}
\addbibresource{bibliography}
\usepackage{bibentry}
%%% BibTeX %%%
%\usepackage[square,sort&compress,numbers]{natbib}
%\usepackage{bibentry}
%\nobibliography*

% Hyperref for hyperlinks in the pdf file (load this package last)
\usepackage{hyperref}

%%%%%%%%%%%%%%%%%%%%%%%%%%%%%%%%%%%%%%%%%%%%%%%%%%%%%%%%%%%%%%%%%%%%%%%%%%%%

% Consistent notation of names of experiments
\newcommand{\Hera}{\textsc{Hera}}
\newcommand{\Hermes}{\textsc{Hermes}}
\newcommand{\BaBar}{\textsc{BaBar}}
\newcommand{\Qweak}{\texorpdfstring{$Q_{Weak}$}{Qweak}}
\newcommand{\boldQweak}{\texorpdfstring{\boldmath$Q_{Weak}$}{Qweak}}
\newcommand{\Moller}{\texorpdfstring{\textsc{Moller}}{MOLLER}}
\newcommand{\JLab}{\texorpdfstring{\textsc{JLab}}{JLab}}
\newcommand{\HIgS}{HI\texorpdfstring{$\gamma$}{g}S}

% Abbreviations in italics
\newcommand{\ie}{\textit{i.e.}}
\newcommand{\eg}{\textit{e.g.}}
\newcommand{\apriori}{\textit{a priori}}
\newcommand{\etal}{\textit{et al.}}

%%%%%%%%%%%%%%%%%%%%%%%%%%%%%%%%%%%%%%%%%%%%%%%%%%%%%%%%%%%%%%%%%%%%%%%%%%%%

% Opening
\title{Biographical Sketch: Wouter Deconinck}
\author{Wouter Deconinck}

%%%%%%%%%%%%%%%%%%%%%%%%%%%%%%%%%%%%%%%%%%%%%%%%%%%%%%%%%%%%%%%%%%%%%%%%%%%%

\begin{document}

% Your Bio should be divided into the following sections
% (a) Professional Preparation (education):
% Undergrad, Major, Year
% Graduate, Major, Year
% Postdoc, Area, Years-Inclusive
% (b) Appointments:  most recent first.
% (c) Products:  5 related to the proposal, and 5 "Other Significant Products"
% (d) Synergistic Activities (math-enhancing activities that were not
% part of your main job description, like editorial boards and
% conference organizing - any Math-related volunteer work.
% these are often similar to Broader Impacts

% (e) Collaborators & Other Affiliations: (use the following sections)
% list in alphabetical order, and include current affiliations parenthetically
% Collaborators and Co-editors: past 48 months.  If none, write "none"
% Graduate Advisors and Postdoctoral sponsors: (your own, no matter how long ago)
% Thesis advisor and postgraduate scholar-sponsor:  those you have advised
% in the past 5 years.  
% Total number of graduate students advised: X (all time)
% Total number of postdoctoral scholars sponsored: Y (all time)

\section*{Biographical Sketch: Wouter Deconinck}

\begin{tabular}{ll}
Address & College of William \& Mary, Department of Physics \\
        & P.O. Box 8795, Williamsburg, VA 23187 \\
Phone   & (757) 221-3539 \\
Email   & wdeconinck@wm.edu \\
\end{tabular}

%---------------------------------------------------------------------------

\subsubsection*{Professional preparation}

\textbf{Undergraduate institutions} \\
University of Gent (Belgium), M.S. in Engineering Physics, July 2003 \\
%
\textbf{Graduate institutions} \\
University of Michigan, Ph.D. in Physics, April 2008 \\
%
\textbf{Postdoctoral institutions} \\
Deutsches Elektronen-Synchrotron (DESY, Germany), hadronic physics, January--May 2008 \\
Massachusetts Institute of Technology, electroweak precision measurements, June 2008--June 2010

%---------------------------------------------------------------------------

\subsubsection*{Appointments}

\textbf{Assistant professor,} July 2010--present \\
Department of Physics, College of William \& Mary, Williamsburg, VA \\
%
\textbf{Postdoctoral associate,} June 2008--June 2010 \\
Laboratory for Nuclear Science, Massachusetts Institute of Technology, Cambridge, MA \\
%
\textbf{Research fellow} (competitive selection), January--May 2008 \\
Deutsches Elektronen-Synchrotron (DESY), Germany \\
%
\textbf{Research assistant,} July 2003--December 2007 \\
Department of Physics, University of Michigan, Ann Arbor, MI

%---------------------------------------------------------------------------

\subsubsection*{Products closely related to the proposed project}

No products closely related to the proposed project.

%---------------------------------------------------------------------------

\subsubsection*{Other significant products\footnotemark{}}
\footnotetext{Due to the size of nuclear and high-energy physics collaborations it is not possible to list all collaborators and affiliations within the constraints of this document.  The full author lists are provided in the references.}
\begin{itemize}
 \setlength{\itemsep}{0pt}
 \setlength{\parskip}{0pt}
 \setlength{\parsep}{0pt}
 \item D.~Androic \etal~
  \newblock First determination of the weak charge of the proton.
  \newblock \emph{Phys. Rev. Lett.}, 111:\penalty0 141803, Oct 2013.
 \item D.~Wang \etal~
  \newblock Measurements of parity-violating asymmetries in electron-deuteron scattering in the nucleon resonance region.
  \newblock \emph{Phys. Rev. Lett.}, 111:\penalty0 082501, Aug 2013.
 \item S.~Abrahamyan \etal~
  \newblock Measurement of the neutron radius of $^{208}\mathrm{Pb}$ through parity violation in electron scattering.
  \newblock \emph{Phys. Rev. Lett.}, 108:\penalty0 112502, Mar 2012a.
 \item Z.~Ahmed \etal~
  \newblock New precision limit on the strange vector form factors of the proton.
  \newblock \emph{Phys. Rev. Lett.}, 108:\penalty0 102001, Mar 2012.
 \item S.~{Abeyratne} \etal~
  \newblock {Science Requirements and Conceptual Design for a Polarized Medium Energy Electron-Ion Collider at Jefferson Lab}.
  \newblock 2012. \label{eic}
\end{itemize}

%---------------------------------------------------------------------------

\subsubsection*{Synergistic activities}
\begin{itemize}
 \setlength{\itemsep}{0pt}
 \setlength{\parskip}{0pt}
 \setlength{\parsep}{0pt}

 \item {\bf Founder and director of the Small Hall Makerspace (2013--present):} The Small Hall Makerspace is a location in the William \& Mary Physics Department where students, faculty and staff with common interests in technology, design and innovation can collaborate. The Small Hall Makerspace is an open community lab incorporating a machine shop and several workshops and studios where makers can come together to build and make things.
 
 \item {\bf W\&M Physics REU Coordinator (2011--present):} The PI has been the coordinator of the long-running physics REU program at W\&M since 2011. Under his tenure the program has revitalized and broadened participation by underrepresented groups.

\end{itemize}

%---------------------------------------------------------------------------

\subsubsection*{Collaborators}
\begin{itemize}
 \setlength{\itemsep}{0pt}
 \setlength{\parskip}{0pt}
 \setlength{\parsep}{0pt}
 \item Postdoctoral collaborators: A.~Bernstein (MIT), D.~Hornidge (Mount Allison University).
 \item Collaborators at Jefferson Lab:
  \begin{itemize}
   \setlength{\itemsep}{0pt}
   \setlength{\parskip}{0pt}
   \setlength{\parsep}{0pt}
   \item \Qweak: \fullcite{PhysRevLett.111.141803}
   \item Hall A: \fullcite{PhysRevLett.111.082501}
   \item HAPPEx: \fullcite{PhysRevLett.108.102001}
   \item PREx: \fullcite{PhysRevLett.108.112502}
   \item \Moller: \fullcite{Proposal:Moller}
   \item SoLID PV-DIS: \fullcite{Proposal:SoLID-PVDIS}
  \end{itemize}
 \item Collaborators at DESY, Germany: HERMES: \fullcite{PhysRevD.87.074029}.
 \item Collaborators at University of Mainz, Germany: K.~Aulenbacher, P.~Aguar Bartolome, V.~Tioukine; A4: \fullcite{PhysRevLett.102.151803}.
\end{itemize}

\subsubsection*{Advisers}
\begin{itemize}
 \setlength{\itemsep}{0pt}
 \setlength{\parskip}{0pt}
 \setlength{\parsep}{0pt}
 \item Postdoctoral sponsors (2): Prof.~Stanley~B.~Kowalski (MIT),  Dr.~Wolf-Dieter~Nowak (DESY)
 \item Graduate advisor (1): Prof. Wolfgang B.~Lorenzon (University of Michigan)
\end{itemize}

\subsubsection*{Research advisees}
\begin{itemize}
\setlength{\itemsep}{0pt}
\setlength{\parskip}{0pt}
\setlength{\parsep}{0pt}
\item Ph.D.~students supervised as an assistant professor: 5; J.~C.~Cornejo (now at Carnegie Mellon University); K.~Bartlett, V.~Gray (in progress at W\&M under my supervision); M.~Beebe, D.~Getts (in progress at W\&M).
% Juan Carlos Cornejo (2010, Compton photon detector)
% Valerie Gray (2012, Qweak tracking and simulations)
% Melissa Beebe (2013, MOLLER simulations)
% Kurtis Bartlett (2013, Qweak simulations)
% Darren Getts (2014, Bubble chamber)
%
\item Ph.D.~students supervised as a postdoctoral researcher: 2.
% Amrendra Narayan (2008--2010, electron detector)
% Jie Pan (2009, momentum reconstruction)
%
\item Undergraduate students supervised in past five years: 25.
% Marcus Hendricks (summer 2009, Qweak tracking, 2d event display)
% Derek Jones (summer 2010, Qweak tracking, 2d event display)
% Juan Carlo Cornejo (summer 2010, Qweak tracking, 3d event display)
% David Zou (summer 2010, Qweak tracking, on/offline tracking GUI)
% Zach Addison (summer 2010, Qweak tracking, 3d event display)
% Andrew Kubera (summer 2010, Moller polarimeter, C++ port of Fortran analyzer)
% David Specht (spring/summer 2011, Qweak beam corrections)
% Rachel Taverner (spring 2011, Qweak beam corrections, summer/fall 2013, spring 2014, Qweak track reconstruction)
% Quinn Hailes (summer 2012, Qweak track reconstruction)
% Jonathan Rigby (fall 2012, Qweak track reconstruction)
% Christopher Haufe (spring/summer/fall 2013, summer 2015, MOLLER simulations)
% Jack Anderson (spring/summer/fall 2013, SoLID simulations)
% Adora Smith (summer 2013, GENIE)
% Nathan Miles (summer 2013, GENIE)
% Marcus Starman (fall 2013, MOLLER simulations)
% Rachel Hyneman (fall 2013, Compton simulations)
% Melissa Guidry (fall 2013, Qweak simulations)
% Karen Ficenec (summer 2015, 3D printing of scintillators)
% Jacob McCormick (summer 2015-fall 2015, 3D printing of scintillators)
% Jacob Elledge (summer 2015, hyperon asymmetries in Qweak)
% Oscar Deaver (summer 2014-spring 2015, multianode PMT testing)

\end{itemize}

%%---------------------------------------------------------------------------%%

% Bibliography
\newpage
\setcounter{page}{1}
%%% BibTeX %%%
%\bibliographystyle{unsrtnat}
%\bibliography{bibliography}
%%% BibLaTeX %%%
%\printbibliography

\end{document}
